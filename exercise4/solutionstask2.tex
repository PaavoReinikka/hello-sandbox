%The paper size, font size and document type are defined in the following
\documentclass[a4paper,12pt]{article}
%\documentclass[a4paper,finnish,12pt]{article}

%Uncomment the following line, if you write in Finnish (special characters)
%\usepackage[utf8]{inputenc}
%if it doesn't work, try
%\usepackage[latin1]{inputenc}

%\usepackage[finnish]{babel}
\usepackage[english]{babel}

%useful special symbols:
\usepackage{amssymb}
\usepackage{latexsym}
\usepackage{amsmath}
\usepackage{amsthm}

%a useful package if you write url addresses:
\usepackage{url}

%a package for figures:
\usepackage[dvips]{color}
\usepackage{epsfig}

%Bibliography style. The alpha style generates references with 
%first letters and year. If you prefer numbers, use style plain.
\bibliographystyle{alpha}


%Create your own environments
\newtheorem{definition}{Definition}
\newtheorem{example}{Example}
\newtheorem{theorem}{Theorem}

\definecolor{red}{rgb}{1.0,0,0}
\definecolor{green}{rgb}{0,1.0,0}
\definecolor{blue}{rgb}{0,0,1.0}
\def\Red{\color{red}}


\newcommand{\iidr}{\ensuremath{\mathit{iidr}}}
\newcommand{\xbf}{\ensuremath{\mathbf{x}}}
\newcommand{\ybf}{\ensuremath{\mathbf{y}}}
\newcommand{\Pset}{\ensuremath{\mathbf{P}}}
\newcommand{\Qset}{\ensuremath{\mathbf{Q}}}
\newcommand{\Kbf}{\ensuremath{\mathbf{K}}}




%If you want to remove the space before paragraphs uncomment the following.
%Remember then to leave an empty line between paragraphs! 
%\setlength{\parindent}{0pt}

%Uncomment the following, if you don't want the date to be printed
%\date{}

\begin{document}

\section*{Recommender system task solutions}


{\bf Table \ref{movier} presents movie ratings by 6 
users on 6 movies. The latex source of the table is available on the course 
page (mratingstable.tex). The ratings are between 1 (didn't like at all) to 5 
(fantastic movie) and 0 means a missing rating (the user hasn't watched 
the movie). The users are notated $u1,\hdots,u6$ and movies $m1,\hdots,m6$. 
The task is to apply recommender systems for rating  prediction using 
neighbourhood-based collaborative filtering (Aggarwal 18.5.2 and an 
example in the lecture).}

\begin{itemize}
\item[a)] {\bf  Calculate mean ratings per user. Use all non-missing ratings in the
calculation.}\\
The row means are\\
$\mu(u1)=2.000$\\
$\mu(u2)=3.000$\\
$\mu(u3)=3.000$\\
$\mu(u4)=4.000$\\
$\mu(u5)=3.600$\\
$\mu(u6)=3.333$

\item[b)] {\bf Calculate required pairwise similarities between
  users\footnote{Note: similarity between $u2$ and $u3$ is not needed,
    so 14 similarities.} using a modified Pearson correlation $r$
  (``Pearson'' in Aggarwal Equation 18.2). Use the mean values
  calculated in part a. Remember that the correlation is calculated
  only over co-rated movies.}\\


User-user similarities (number of common ratings in parenthesis):

\begin{center}
\begin{tabular}{|r|r|r|r|r|r|r|}
\hline
&u1&u2&u3&u4&u5&u6\\
\hline
u1&1.000 (5)&0.816 (5)&0.707 (5)&1.000 (3)&-0.811 (4)&-0.721 (3)\\
u2&0.816 (5)&1.000 (6)&0.000 (6)&1.000 (4)&-0.559 (5)&-0.721 (3)\\
u3&0.707 (5)&0.000 (6)&1.000 (6)&0.316 (4)&-0.589 (5)&-0.557 (3)\\
u4&1.000 (3)&1.000 (4)&0.316 (4)&1.000 (4)&-0.684 (3)&-0.371 (2)\\
u5&-0.811 (4)&-0.559 (5)&-0.589 (5)&-0.684 (3)&1.000 (5)&0.905 (2)\\
u6&-0.721 (3)&-0.721 (3)&-0.557 (3)&-0.371 (2)&0.905 (2)&1.000 (3)\\
\hline
\end{tabular}
\end{center}


\item[c)] {\bf Predict missing ratings using two nearest neighbours
  ($K=2$) and an extra requirement that the similarity is $r\geq
  0.5$. Tell if the movie is recommended to the user (if the user
  would like it more than average).\\
Report if some prediction cannot be made (not enough 
sufficiently similar neighbours with required ratings).}

$u1$: nearest neighbours are $u4$ and $u2$ (and both $r\geq 
0.5$). Predicted rating to $m5$ is 3.000 $>\mu(u1)$, so recommend.\\

$u4$: nearest neighbours $u1$ and $u2$ (and $r\geq 0.5$). 
For $m1$ the prediction is 5.000 $>\mu(u4)$, so recommend.\\
For $m6$ the prediction is 3.500 $<\mu(u4)$, don't recommend.\\

$u5$: Only $u6$ sufficiently close neighbour, predictions cannot be made.\\
$u6$: Only $u5$ sufficiently close neighbour, predictions cannot be made.\\

(Extra note: $u5$ and $u6$ have only 2 common ratings, so the $r$
value is not very reliable.)\\


\item[d)] {\bf Consider the item-based way of predicting the missing ratings of
  movies $m3$ and $m4$ with adjusted cosine similarity, as suggested
  in Aggarwal 18.5.2.2. Why it is not a good solution here? Suggest an
  alternative item-based solution that could be used instead (no need
  to calculate the actual predictions).}\\

Aggarwal suggest to choose the most similar items with adjusted cosine 
similarity. However, it doesn't work here at all.  
Ratings for $m3$ and $m4$ are indentical (if neither is missing), i.e., all
users have liked them equally much. This means that they should have
maximal similarity.  However, adjusted cos-sim evaluates similarity as
0 (very dissimilar). The reason is that users have given average
ratings to movies $m3$ and $m4$ and subtracting the user means
(mean-centering) produces zero vectors, whose dot product is zero.

One solution is to use Pearson correlation coefficient for similarity
between items. It gets value 1.0, i.e., perfect similarity. (Extra
note: here the mean values of two items' ratings are the same, so
using a modified Pearson doesn't cause any difference. If this was 
not the case, the similarity could be smaller.)


\end{itemize}

\newpage
\begin{table}[!h]
\begin{center}
\caption{Movie ratings (scale 1--5) by 6 users ($u1$--$u6$) on 6 movies 
($m1$--$m6$). Special value 0 means a missing rating.}
\label{movier}
\begin{tabular}{|l|l|l|l|l|l|l|}
\hline
&$m1$&$m2$&$m3$&$m4$&$m5$&$m6$\\
\hline
u1&3&1&2&2&0&2\\
u2&4&2&3&3&4&2\\
u3&4&1&3&3&2&5\\
u4&0&3&4&4&5&0\\
u5&2&5&5&0&3&3\\
u6&1&4&0&5&0&0\\
\hline
\end{tabular}
\end{center}
\end{table}




\bibliographystyle{abbrv}
%add bibtex collection file here
%\bibliography{} 
\end{document}
