%The paper size, font size and document type are defined in the following
\documentclass[a4paper,12pt]{article}
%\documentclass[a4paper,finnish,12pt]{article}

%Uncomment the following line, if you write in Finnish (special characters)
%\usepackage[utf8]{inputenc}
%if it doesn't work, try
%\usepackage[latin1]{inputenc}

%\usepackage[finnish]{babel}
\usepackage[english]{babel}

%useful special symbols:
\usepackage{amssymb}
\usepackage{latexsym}
\usepackage{amsmath}
\usepackage{amsthm}

%a useful package if you write url addresses:
\usepackage{url}

%a package for figures:
\usepackage[dvips]{color}
\usepackage{epsfig}

%Bibliography style. The alpha style generates references with 
%first letters and year. If you prefer numbers, use style plain.
\bibliographystyle{alpha}


%Create your own environments
\newtheorem{definition}{Definition}
\newtheorem{example}{Example}
\newtheorem{theorem}{Theorem}



\definecolor{red}{rgb}{1.0,0,0}
\definecolor{green}{rgb}{0,1.0,0}
\definecolor{blue}{rgb}{0,0,1.0}
\def\Red{\color{red}}
\newcommand{\xbf}{\ensuremath{\mathbf{x}}}
\newcommand{\ybf}{\ensuremath{\mathbf{y}}}
\newcommand{\zbf}{\ensuremath{\mathbf{z}}}
\newcommand{\cbf}{\ensuremath{\mathbf{c}}}
\newcommand{\Sim}{\ensuremath{\mathit{sim}}}
\newcommand{\Wmatr}{\ensuremath{\mathbf{W}}}
\newcommand{\Lmatr}{\ensuremath{\mathbf{L}}}
\newcommand{\Lambdamatr}{\ensuremath{\mathbf{\Lambda}}}
\newcommand{\WSI}{\ensuremath{\mathit{WSI}}}
\newcommand{\BMI}{\ensuremath{\mathit{BMI}}}
\newcommand{\NMI}{\ensuremath{\mathit{NMI}}}
\newcommand{\Xset}{\ensuremath{\mathbf{X}}}
\newcommand{\Yset}{\ensuremath{\mathbf{Y}}}
\newcommand{\Zset}{\ensuremath{\mathbf{Z}}}
\newcommand{\Qset}{\ensuremath{\mathbf{Q}}}
\newcommand{\Rset}{\ensuremath{\mathbf{R}}}
\newcommand{\Cisc}{\ensuremath{C{=}c}}
\newcommand{\MI}{\ensuremath{\mathit{MI}}}
\newcommand{\fr}{\ensuremath{\mathit{fr}}}


\newcommand{\Udist}{\ensuremath{\mathit{Udist}}}
\newcommand{\Mdist}{\ensuremath{\mathit{Mdist}}}
\newcommand{\MCS}{\ensuremath{\mathit{MCS}}}


%If you want to remove the space before paragraphs uncomment the following.
%Remember then to leave an empty line between paragraphs! 
%\setlength{\parindent}{0pt}

%Uncomment the following, if you don't want the date to be printed
%\date{}

\begin{document}

\section*{CS-E4650 Methods of Data mining}
\textbf{\large Exercise 4 / Autumn 2023}

%exercise round number
\setcounter{section}{4}


%%%%%%%%%%%%%%%%%%%%%%%%%%%%%%%%%%%%%%%%%%%%%%%%%%%%%%%%%%%%%%%%%%%%%%%%%
\subsection{PageRank and HITS}

{\em Learning goal: The idea of PageRank and HITS algorithms.}\\

Figure \ref{webgraph} shows the linkage structure of web pages and
Table \ref{webcont} lists the keywords that occur in the pages. The
task is to evaluate PageRank and hubs and authority values of pages
given a query. In this task, you can use any of the existing PageRank and
HITS (simulation) tools (you can find also online calculators). Note
that different tools may use different initialization or scaling but
the top results should be the same. If the tool allows you to adjust
the teleportation probability, use 0.10 or 0.15.

\begin{figure}[!h]
\begin{center}
\includegraphics[width=0.4\textwidth]{pagegraph.eps}
\end{center}
\caption{Linkage structure of web pages.}
\label{webgraph}
\end{figure}

\begin{table}[!h]
\caption{Keywords that occur in pages A--J.}
\label{webcont}
\begin{center}
\begin{tabular}{|l|l|}
\hline
id&keywords\\
\hline
A&authority, page, reputable, source\\
B&hub, page, link, good, source\\
C&PageRank, HITS, ranking, algorithm\\
D&reputable, page, link, PageRank\\
E&reputation, visit, frequency, random, surfer\\
F&random, surfer, trap, dead-end\\
G&PageRank, teleportation, random, surfer, model\\
H&teleportation, travel, planet\\
I&Star Trek, transporter, teleportation\\
J&beam, Scotty, transporter\\
\hline
\end{tabular}
\end{center}
\end{table}

\begin{itemize}
\item[a)] Evaluate PageRank values for all pages. What would be the most reputable sources containing query words i) ``PageRank'' or ii) ``teleportation''?
\item[b] Construct the HITS graph (base set and its edges) for query ``Page\-Rank''. Include pages in the root set, all pages pointed by the root pages and all pages pointing to the root pages. Then calculate the hubs and authority values. Which page is the best authority on the topic? What about the best hub?
\item[c)] Repeat b) for query ``teleportation''. 
\item[d)] Compare the results with PageRank and HITS. Do they agree on the most reputable sources?
\end{itemize}


%%%%%%%%%%%%%%%%%%%%%%%%%%%%%%%%%%%%%%%%%%%%%%%%%%%%%%%%%%%%%%%%%%%%%%%%%
\subsection{Collaborative filtering for movie recommendations}

{\em Learning goal: How to use neighbourhood-based collaborative filtering in recommender systems; problems of adjusted cosine similarity.} 

Table \ref{movier} presents movie ratings by 6 
users on 6 movies. The latex source of the table is available on the course 
page (mratingstable.tex). The ratings are between 1 (didn't like at all) to 5 
(fantastic movie) and 0 means a missing rating (the user hasn't watched 
the movie). The users are notated $u1,\hdots,u6$ and movies $m1,\hdots,m6$. 
The task is to apply recommender systems for rating  prediction using 
neighbourhood-based collaborative filtering (see Aggarwal 18.5.2 and an 
example in the lecture). 

\begin{itemize}
\item[a)] Calculate mean ratings per user. Use all non-missing ratings in the
calculation. These are needed in parts b) and c).

\item[b)] Calculate required pairwise similarities between
  users\footnote{Note: similarity between $u2$ and $u3$ is not needed,
    so 14 similarities.} using a modified Pearson correlation $r$
  (``Pearson'' in Aggarwal Equation 18.12). Use the mean values
  calculated in part a. Remember that the correlation is calculated
  only over co-rated movies.

\item[c)] Predict missing ratings using two nearest neighbours
  ($K=2$) and an extra requirement that the similarity is $r\geq
  0.5$. Tell if the movie is recommended to the user (if the user
  would like it more than average).

Report if some prediction cannot be made (not enough 
sufficiently similar neighbours with required ratings).

\item[d)] Consider the item-based way of predicting the missing ratings of
  movies $m3$ and $m4$ with adjusted cosine similarity, as suggested
  in Aggarwal 18.5.2.2. Why it is not a good solution here? Suggest an
  alternative item-based solution that could be used instead (no need
  to calculate the actual predictions).

\end{itemize}

\begin{table}[!h]
\begin{center}
\caption{Movie ratings (scale 1--5) by 6 users ($u1$--$u6$) on 6 movies 
($m1$--$m6$). Special value 0 means a missing rating.}
\label{movier}
\begin{tabular}{|l|l|l|l|l|l|l|}
\hline
&$m1$&$m2$&$m3$&$m4$&$m5$&$m6$\\
\hline
u1&3&1&2&2&0&2\\
u2&4&2&3&3&4&2\\
u3&4&1&3&3&2&5\\
u4&0&3&4&4&5&0\\
u5&2&5&5&0&3&3\\
u6&1&4&0&5&0&0\\
\hline
\end{tabular}
\end{center}
\end{table}

%%%%%%%%%%%%%%%%%%%%%%%%%%%%%%%%%%%%%%%%%%%%%%%%%%%%%%%%%%%%%%%%%%%%%%%%%
\subsection{Distances between molecular structures}

{\em Learning goal: The concept of maximum common subgraph (MCS) and related distance measures.}\\

Figure \ref{comp} shows an example of four molecular graph structures:
Niacin (vitamin B1), Nicotine (active ingredient in tobacco), psilocin
(active ingredient in ``magic mushrooms''), and proline (amino acid).
The node labels correspond to atoms (carbon, oxygen or
nitrogen)\footnote{For simplicity hydrogen atoms and double bonds
  between atoms are not presented.}.

\begin{figure}[!h]
\begin{center}
\includegraphics[width=\textwidth]{newchem.eps}
\caption{Four graphs corresponding to molecular structures.}
\label{comp}
\end{center}
\end{figure}

\begin{itemize}
\item[a)] Determine the nearest neighbour for each molecule using Union-normalized MCS distance ($\Udist$ in slides, see Aggarwal Eq.~17.2).
\item[b)] Determine the nearest neighbour for each molecule using Max-normalized MCS distance ($\Mdist$ in slides, see Aggarwal Eq.~17.3).
\item[c)] Under which conditions are $\Udist$ and $\Mdist$ equivalent? I.e., give conditions related to some graphs $G_1$, $G_2$ and $\MCS(G_1,G_2)$ such that $\Udist(G_1,G_2)=\Mdist(G_1,G_2)$.
\end{itemize}


\newpage
%%%%%%%%%%%%%%%%%%%%%%%%%%%%%%%%%%%%%%%%%%%%%%%%%%%%%%%%%%%%%%%%%%%%%%%%%
\subsection{Homework: This will be added later}

This task is homework that is done in groups of 2--3 students.  Note
that you cannot do the task alone or in a larger group, so it is
recommended to search a group now. The TAs can help to find collaborators.

%%%%%%%%%%%%%%%%%%%%%%%%%%%%%%%%%%%%%%%%%%%%%%%%%%%%%%%%%%%%%%%%%%%%%%%%%




\bibliographystyle{abbrv}
%add bibtex collection file here
%\bibliography{} 
\end{document}
